%!TEX root=main.tex
\subsection{Schädigung}
% https://de.wikipedia.org/wiki/Sehschw%C3%A4che
Schäden am visuellen System werden grob anhand der betroffenen Einzelteile kategorisiert.

\subsubsection{Optische Fehlsichtigkeit}
% https://de.wikipedia.org/wiki/Ametropie
\begin{quote}
\enquote{Beim \textit{kurzsichtigen} Auge liegt das Bild ferner Gegenstände \textit{vor} der Netzhaut. Zur Korrektur wird daher eine Zerstreuungslinse verwendet.} \cite[S. 236]{physik1}
\end{quote}
% TODO Bildbeispiel!

\begin{quote}
\enquote{Beim \textit{weitsichtigen} Auge liegt das Bild ferner Gegenstände \textit{hinter} der Netzhaut. Die Augenlinse muss also durch eine Sammellinse unterstützt werden.} \cite[S. 237]{physik1}
\end{quote}
% TODO Bildbeispiel!

Die optische Fehlsichtigkeit steht nicht in Zusammenhang mit der Altersweitsichtigkeit, welche durch ein Nachlassen der Elastizität am Auge zustande kommt, wodurch Betroffenen kein scharfes sehen in der Nähe mehr möglich ist.
