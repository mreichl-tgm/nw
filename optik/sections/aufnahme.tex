\section[Optische Informationsaufnahme\hfill Chemische und physikalische Technologie]{Optische Informationsaufnahme\\{\normalsize Chemische und physikalische Technologie}}
\subsection{Kamera}
Die Kamera orientiert sich in ihrer Konstruktion am menschlichen Auge (Abschnitt \ref{sec:auge}). 

Das Objektiv (die Linse) bündelt einfallendes Licht, welches von einem lichtempfindlichen Aufnahmemedium (der Netzhaut) als auf dem Kopf stehendes gespiegeltes Abbild der Umgebung aufgezeichnet wird. Im Gegensatz zum Auge besteht das Objektiv dabei aus mehreren Linsen, welche gemeinsam als eine Sammellinse wirken.

% TODO: Satz richtigstellen
Digitalkameras nutzen zur Aufnahme Sensoren, die das Licht als elektronisches Signal interpretieren, um ein digitales Bild zu erzeugen.