%!TEX root=main.tex
\section{Photon}
Das Photon ist ein Elementarteilchen und Übermittler der elektromagnetischen Wechselwirkung. Nach Max Planck bilden Photonen die Energiepakete elektromagnetischer Strahlung, weshalb sie auch Lichtquanten genannt werden. 

Das plancksche Wirkungsquantum ($h$) ist das Verhältnis von Energie ($E$) und Frequenz ($f$) eines Photons.
$$E=hf$$
\begin{vardef}
	\addvardef{$E$}{Energie}
	\addvardef{$h$}{Plancksches Wirkungsquantum}
	\addvardef{$f$}{Frequenz}
\end{vardef}

Der Relativitätstheorie nach wirkt sich jede Änderung ($\Delta$) der Energie ($E$) auch auf die Masse ($m$) aus.
$${\Delta}m = \frac{{\Delta}E}{c^2} = \frac{hf}{c^2}$$
\begin{vardef}
	\addvardef{${\Delta}m$}{Masseänderung}
	\addvardef{$c$}{Lichtgeschwindigkeit}
\end{vardef}

Diese Masse \enquote{bewirkt, dass Photonen von Gravitationsfeldern abgelenkt werden und dabei Energie verlieren oder gewinnen können.} \cite{ulm:photon} Über diese kann zudem der Impuls eines Photons berechnet werden.
$$$$

Eine eigene Masse hat ein Photon nicht, da es sich stets mit Lichtgeschwindigkeit fortbewegt. Diese Eigenschaft kann über den Impuls ($\vec{p}$) bewiesen werden.
$$m = \frac{\vec{p}}{\vec{v}}\sqrt{1-\frac{v^2}{c^2}} = 0$$

$$p=mc=\frac{hf}{c}=\frac{h}{\lambda}$$

% \subsection{Photoeffekt}\label{subsec:photoeffekt}
