%!TEX root=main.tex
\section{Photon}
Das Photon ist ein Elementarteilchen und Übermittler der elektromagnetischen Wechselwirkung. Nach Max Planck bilden Photonen die Energiepakete elektromagnetischer Strahlung, weshalb sie auch Lichtquanten genannt werden. Das plancksche Wirkungsquantum ($h$) ist das Verhältnis von Energie ($E$) und Frequenz ($f$) eines Photons.
$$E=hf$$
\begin{vardef}
	\addvardef{$E$}{Energie}
	\addvardef{$h$}{Plancksches Wirkungsquantum}
	\addvardef{$f$}{Frequenz}
\end{vardef}

Der Relativitätstheorie nach wirkt sich jede Änderung ($\Delta$) der Energie ($E$) auch auf die Masse ($m$) aus.
$${\Delta}m = \frac{{\Delta}E}{c^2} = \frac{hf}{c^2}$$
\begin{vardef}
	\addvardef{${\Delta}m$}{Masseänderung}
	\addvardef{${\Delta}E$}{Energieänderung}
	\addvardef{$c$}{Lichtgeschwindigkeit}
\end{vardef}

Diese Masse \enquote{bewirkt, dass Photonen von Gravitationsfeldern abgelenkt werden und dabei Energie verlieren oder gewinnen können.} \cite{ulm:photon}

Eine eigene Masse hat ein Photon nicht, da es sich stets mit Lichtgeschwindigkeit fortbewegt. Über den Impuls kann diese Eigenschaft mathematisch bewiesen werden.
$$m = \frac{\vec{p}}{\vec{v}}\sqrt{1-\frac{v^2}{c^2}} = 0$$
\begin{vardef}
	\addvardef{$m$}{Masse}
	\addvardef{$\vec{p}$}{Impulsvektor}
	\addvardef{$\vec{v}$}{Geschwindigkeitsvektor}
	\addvardef{$v$}{Geschwindigkeit}
\end{vardef}

Umgekehrt lässt sich diese Definition nicht zur Berechnung des Impulses für Objekte ohne Masse anwenden, weshalb dafür auf die Energie-Impuls-Beziehung zurückgegriffen wird.
% TODO: Erklärung zu Energie-Impuls-Beziehung
$$E^2-p^2c^2 = m^2c^4 \quad E=pc$$
$$p=\frac{E}{c}=\frac{hf}{c}=\frac{h}{\lambda}$$
\begin{vardef}
	\addvardef{$p$}{Impuls}
	\addvardef{$\lambda$}{Wellenlänge}
\end{vardef}
% TODO: Herleitung lambda

% TODO: \subsection{Photoeffekt}\label{subsec:photoeffekt}