%!TEX root=main.tex
\section{Geschichte}
\begin{quote}
\enquote{Wie bei allen komplizierten Naturphänomenen versuchten die Physiker auch beim Licht eine Beschreibung durch möglichst einfache, mechanische Modelle. Im Altertum glaubte man an \enquote{Sehstrahlen}, die vom Auge ausgehen und die Außenwelt abtasten. Genauere Vorstellungen vom Wesen des Lichtes kamen um 1670 auf.} \cite[S. 221]{physik1}
\end{quote}

Innerhalb kurzer Zeit wurden zwei konkurrierende Theorien aufgestellt:
\begin{outline}
	\1 Isaac Newton vertrat die \textit{Teilchentheorie}, nach welcher das Licht aus kleinsten Teilchen, sogenannten Korpuskeln besteht. Diese würden von Körpern reflektiert und vom menschlichen Auge wahrgenommen. Die Theorie wurde als Korpuskeltheorie bekannt.
	
	\1 Christian Huygens dagegen stellte eine \textit{Wellentheorie} auf, wobei er die Ausbreitung von Licht mit jener von Schall- oder Wasserwellen verglich. Anhand des Huygensschen Prinzips war auch ihm eine Erklärung der bekannten Eigenschaften des Lichtes möglich.
\end{outline}

Erst 1802 gelang Augenarzt und Physiker Thomas Young der Beweis zur Wellentheorie, durch die Entdeckung von \gls{interferenz} an Lichtwellen.

James Maxwell stellte dazu 1864 die \textit{\gls{maxwelltheorie}} auf, welche später durch Heinrich Hertz bestätigt und von Hendrik Lorentz erweitert wurde.

Das sichtbare Licht stellt nur einen kleinen Teil aller elektromagnetischen Wellen dar.

\subsection{Dualismus}
Im Jahr 1888 beobachtete Hans Hellwachs bei der Bestrahlung einer Metallplatte mit UV-Licht, einen Verlust an negativer Ladung. Dieses Auslösen von Elektronen wurde als Photoeffekt~%(Abschnitt \ref{subsec:photoeffekt})
bekannt und stand mit der Wellentheorie des Lichtes in Konflikt.

Albert Einstein wagte 1905 eine Erklärung, indem er auf die Hypothese Max Plancks zurückgriff, dass Körper Strahlung nur in Paketen transportieren. Er bestätigte damit die Teilchennatur des Lichtes und erhielt 1921 den Nobelpreis in Physik.

Damit stellte sich natürlich die Frage, wie diese Erkenntnisse mit der Wellennatur des Lichtes vereinbar sind. Physiker Max Born formulierte dazu auf Basis des \glspl{doppelspalt} folgenden Zusammenhang:
\begin{quote}
\enquote{Die Lichtintensität (das Quadrat der Amplitude) ist proportional zur Wahrscheinlichkeit, in einem bestimmten Raumbereich ein Photon anzutreffen.} \cite[S. 183]{physik2}
\end{quote}