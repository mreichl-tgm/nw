%!TEX root=main.tex
\newglossaryentry{interferenz}{
	name={Interferenz},
	description={Interferenz beschreibt die Amplitudenänderung bei der Überlagerung mehrerer Wellen. Sie tritt bei allen Arten von Wellen auf, wie etwa bei Schall-, Licht- und Materiewellen}
}

\newglossaryentry{maxwelltheorie}{
	name={elektromagnetische Lichttheorie},
	description={\begin{otherlanguage}{english}\enquote{This velocity is so nearly that of light, that it seems we have strong reason to conclude that light itself (including radiant heat, and other radiations if any) is an electromagnetic disturbance in the form of waves propagated through the electromagnetic field according to electromagnetic laws.} \cite{maxwell}\end{otherlanguage}}
}

\newglossaryentry{doppelspalt}{
	name={Doppelspaltexperiment},
	plural={Doppelspaltexperiments},
	description={Beim Doppelspaltexperiment wird kohärentes durch zwei beieinanderliegende Spalte geschickt. An beiden Spalten entstehen neue Elementarwellen, welche sich überlagern und beim Auftreffen als Interferenzmuster aus hellen und dunklen Streifen dargestellt werden.}
}