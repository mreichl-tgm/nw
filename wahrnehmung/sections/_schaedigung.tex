% TODO: \section{Schädigung von Augen und Ohren}
% TODO: \subsection{Schädigung der Augen}
% TODO: Bildschirme (Blaulichtfilter)
% TODO: \subsubsection{Schutz der Augen}
% TODO: \newpage

\subsection{Schädigung der Ohren}
Temporäre Gehörschäden können bereits nach wenigen Stunden unter einer Lärmintensität von mehr als 85 \gls{ac:db} oder einer Frequenz von 4 \gls{ac:khz}\footnote{Ab 4 \gls{ac:khz} wird die unterste Windung der Hörschnecke belastet, was zu einer Schädigung der Haarzellen führt.} zustande kommen. Diese werden meist als Hörermüdungen bezeichnet.

\subsubsection{Lärmschwerhörigkeit}
Lärmschwerhörigkeit beschreibt die Beeinträchtigung des Hörvermögens durch regelmäßige Lärmbelastung. Diese führt in weiterer Folge dazu, dass etwa Sprache und niedrige Frequenzen nicht mehr oder nur eingeschränkt verstanden werden können.

Die Entwicklung einer Lärmschwerhörigkeit ist abhängig von der
\begin{outline}
	\1 Dauer der Lärmexposition,
	\1 Pegel und Frequenz des einwirkenden Lärms,
	\1 Lärmpausen während der Exposition.
\end{outline}

~\\
Im Jahr 2015 ging die Lärmschwerhörigkeit, auf Basis einer Studie der \gls{ac:auva} \cite{auva:berufskrankheit}, als häufigste Berufskrankheit hervor, nachdem dieser 564 aus 1.093 Berufskrankheiten von Erwerbstätigen zugeordnet wurden.

\subsubsection{Knalltrauma}
Entgegen einer Lärmschwerhörigkeit wird beim Knalltrauma das Innenohr, durch das Einwirken eines sehr hohen Schalldrucks geschädigt. Meist ist dies bereits bei einer sehr kurzen Exposition von wenigen Millisekunden der Fall. Ein längeres Einwirken führt zur Verletzung des Trommelfells und der Knöchelchen im Mittelohr, welche als Explosionstrauma bezeichnet wird.

Ursachen von Knalltraumata sind etwa Knallkörper, Schüsse, Sprengungen oder Schläge auf das Ohr.

Zu den Symptomen zählen meist
\begin{outline}
\1 Hörverlust,
\1 Ohrgeräusche (Abschnitt \ref{subsubsec:tinnitus}),
\1 Geräuschempfindlichkeit oder
\1 Gleichgewichtsstörungen
\end{outline}

\newpage
\subsubsection{Ohrgeräusche (Tinnitus)}\label{subsubsec:tinnitus}
\enquote{Medizinisch definiert ist der Tinnitus als akustische Wahrnehmung, die ohne entsprechenden akustischen Reiz von außerhalb des Körpers entsteht und keinen Informationsgehalt besitzt. Letzteres grenzt den Tinnitus von akustischen Halluzinationen ab, bei denen die Betroffenen beispielsweise Stimmen hören.} --- \citeauthor{netdoktor:tinnitus},~\citeyear{netdoktor:tinnitus}~\cite{netdoktor:tinnitus}

Prinzipiell werden 2 Formen des Tinnitus unterschieden. Beim selteneren objektiven Tinnitus gibt es eine körpereigene, messbare Schallquelle, während beim subjektiven Tinnitus nur von den Betroffenen, Töne und Geräusche wahrgenommen werden.

Der objektive Tinnitus entsteht zum Beispiel durch
\begin{outline}
\1 Verengungen von Blutgefäßen,
\1 Verkrampfungen von Muskeln in der Nähe des Ohrs oder
\1 Tumore im Mittelohr.
\end{outline}

Die Entstehung des subjektiven Tinnitus wird meist in körperliche und psychische Ursachen unterteilt. Körperlicher Natur sind etwa der \textit{Hörsturz}, \textit{Mittelohrentzündungen}, \textit{Medikamente}, \textit{Drogen} und auch \textit{Verspannungen}. Psychische Ursachen sind vorwiegend \textit{Stress}, \textit{Lärmbelastung} und \textit{allgemeine psychische Belastungen}.

\subsubsection{Schutz der Ohren}
Hörschäden zu vermeiden ist wesentlich leichter, als diese wieder zu beheben. Häufig genannte Möglichkeiten zum Schutz der Ohren sind etwa die
\begin{outline}
\1 \textbf{Reduktion der Lautstärke:} Fernseher, Musik und Spiele müssen nicht laut sein.
\1 \textbf{Reduktion von Lärmquellen:} Auch die Anzahl der verschiedenen Geräuschquellen wirkt sich negativ auf das Gehör aus. Nicht benötigte Geräte können zu diesem Zweck auch ausgeschaltet werden.
\1 \textbf{Anschaffung leiser Geräte:} Beim Kauf eines Geräts auf die Lautstärke eines Produkts zu achten kann im Extremfall sogar Langzeitschäden Vermeiden.
\1 \textbf{Verwendung von Ohrenschützern:} Eine dauerhaft laute Umgebung sollte immer mit Ohrenschützern betreten werden. In einigen Arbeitsbereichen ist das Tragen von diesen sogar gesetzlich vorgeschrieben.
\end{outline}